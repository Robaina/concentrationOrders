\documentclass[12pt]{article}
\usepackage{authblk}
\usepackage{amsmath}
\usepackage{graphicx}
\usepackage{hyperref}

\title{On the ordering of metabolite concentrations in \emph{Escherichia coli}}
\author{Semid\'an Robaina Est\'evez \footnote{Corresponding autor: \href{mailto:semidan.robaina@gmail.com}{semidan.robaina@gmail.com}}}

\affil{Ronin Institute for Independent Scholarship}

\setcounter{Maxaffil}{0}
\renewcommand\Affilfont{\itshape\small}

\date{\vspace{-5ex}}
\graphicspath{{./images/}}

\begin{document}
  \maketitle
  \bibliographystyle{apalike}

  \begin{abstract}
    Metabolic networks at steady state operate under stoichiometric, kinetic and thermodynamic constraints. As a result, metabolite concentrations are not free to vary. Recently, it has been shown that certain metabolites in \emph{Escherichia coli} show a natural ordering at steady state, \emph{i.e.}, there are metabolite pairs for which one metabolite maintains a higher concentration level than the other under different steady states. Here, we provide an explanation to such order relations in a setting were \emph{Escherichia coli}'s cells growth at steady state.
  \end{abstract}

  \section{Main Idea}
  Several studies have shown that certain metabolites maintain an order relation in their concentrations under different steady states of \emph{Escherichia coli} \cite{Bennett2008}. This order relation may be the result of diverse constraints operating at steady state, such as stoichiometric, thermodynamic and growth constraints. In the following, we will derive a theoretical explanation for this ordering of concentrations. We describe the concentration dynamics of the biochemical network with the system,

  \begin{equation}
    \label{eq:1}
    \dot x = Sv(x),
  \end{equation}

  where, $S$ represents the stoichiometric matrix, $x$ the metabolite concentrations and $v(x)$ the metabolic fluxes, which are functions of metabolite concentrations. Now, all reactions in \ref{eq:1} are reversible --- that is, minus the biomass production (pseudo)reaction --- hence, we have that the net flux of a reaction $v_i = v_i^{for} - v_j^{back}$. We assume that cells are growing at steady state, thus the flux through the biomass reaction, $v_{bio} > \gamma v^{max}_{bio}$, with $\gamma \in [0, 1]$, a fraction of the theoretical maximum.

  Further, we can determine which reactions are irreversible with the given constraints. For this, we only need to solve the linear programs

  \begin{align}
    \label{eq:2}
    \begin{aligned}
      &z^{for}_i=\max_{v}\; v^{for}_i
      \\
      &\mathrm{s.t.}
      \\
      & Sv = 0
      \\
      & v_{bio} > \gamma v^{max}_{bio},
    \end{aligned}
    \qquad
    \begin{aligned}
      &z^{back}_i=\max_{v}\; v^{back}_i
      \\
      &\mathrm{s.t.}
      \\
      & Sv = 0
      \\
      & v_{bio} > \gamma v^{max}_{bio},
    \end{aligned}
  \end{align}

  if $z^{for}_{max} > 0$ and $z^{back}_{max} = 0$ then reaction $v_i$ can only proceed in the forward direction under these constraints. Alternatively, if $z^{for}_{max} = 0$ and $z^{back}_{max} > 0$ then reaction $v_i$ can only proceed in the backward direction.

  Thus far, we have determine that certain reactions are irreversible in an escenario where cells grow at steady state. Additionally, thermodynamic constraints impose the following relation

  \begin{equation}
    \label{eq:3}
    \Delta^{\circ} G_i = \Delta^{\circ} G_{r(i)} + RT\log{\left( Q \right)}
  \end{equation}

  which connects the reaction Gibbs energy $\Delta^{\circ} G_i$ with the sum of the energies of formation of all substrates and products involved in the reaction, \emph{i.e.}, $\Delta^{\circ} G_{r(i)} = \sum_{p \in P} \Delta^{\circ} G_{f(s)} - \sum_{s \in S} \Delta^{\circ} G_{f(s)}$ for each substrate $s$ in the set of reaction substrates $S$ and for all products $p$ in $P$, with the reaction quotient

  $$Q = \frac{\displaystyle \prod_{p \in P}{x_p^{n_{p,i}} }}{\displaystyle \prod_{s \in S}{x_s^{n_{s,i}} } },$$ where $x$ represent concentrations and $n$ the stoichiometric coefficients of each metabolite.

  Further, to maintain the irreversibility of a reaction we need $\Delta^{\circ} G_i < 0$, hence the following inequality must be satisfied $-\Delta^{\circ} G_{r(i)} > RT\log{Q}$. After some algebra, we arrive at

  \begin{equation}
    \label{eq:4}
    \frac{1}{Q} > e^{\frac{\Delta^{\circ} G_{r(i)}}{RT}}
  \end{equation}

  which implies

  \begin{equation}
    \label{eq:5}
    \displaystyle \prod_{s \in S}{x_s^{n_{s,i}}} > e^{\frac{\Delta^{\circ} G_{r(i)} }{RT}} \displaystyle \prod_{p \in P}{x_p^{n_{p,i}}.}
  \end{equation}

  Form the inequality in \ref{eq:4} we conclude that, with fixed temperature, the reaction quotient is solely determined by $\Delta^{\circ} G_{r(i)}$, which, we can estimate with the group contribution method. This result implies that for monosubstrate and monoproduct reactions, we can readily predict an order relation in the metabolite concentrations under growth at steady state. Furthermore, for multimolecular reactions, we could establish a system of inequalities that could be solved to obtain the order relations.

  Therefore, this approach can provide an explanation for the observed concentration order relations in \emph{Escherichia coli}.

  \section{Testing saturation of enzymes under growing conditions}
  In reference \cite{Bennett2008} authors observe that most enzymes with measured substrate levels are saturated. This is a very interesting observation because a network of saturated enzymes implies linear dynamics. I propose here a method to test whether $K_M$ values have evolved to guarantee enzyme saturation. We first note that some reactions may proceed irreversibly under growth, and we can find these reactions through the LPs in \ref{eq:2}. Now, provided we have access to $\Delta^{\circ} G_{r(i)}$, we can compute the minimum substrate complex concentration product required for the reaction to proceed in that direction.

  \textbf{NOTE}: Actually, no... I'm missing here the concentration of the products! What I was after is a way to tell whether the minimum required substrate concentration is above the $K_M$ value, but of course, without knowing the actual concentration. I think this is not possible, is it?





  \bibliography{C:/Users/tinta/OneDrive/Documents/Projects/Bibliography/Ordering_of_concentrations}
\end{document}
